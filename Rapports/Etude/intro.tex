\chapter{Introduction}
	
	Dans le cadre de notre formation Génie Électrique à l'INSA Strasbourg, le \textbf{projet GE II} du semestre 8 a pour objectif la conception d'un système électrique comprenant des éléments de conversion statique et de commande électronique programmable. Ce projet sera notamment l'occasion de mettre en œuvre nos connaissances acquises en électronique de puissance, en électronique numérique, en informatique industrielle et en automatique tout au long de notre cursus. 
	
	Le thème retenu cette année consiste en une compétition baptisée "La nuit de la glisse". L'objectif sera de présenter pour cette compétition un \textbf{aéroglisseur radiocommandé} qui devra répondre à un certain nombre d'exigences pour pouvoir être homologué et ainsi participer à l'épreuve finale. En vue de réaliser un tel système, nous partirons d'un cahier des charges succinct pour mener une analyse fonctionnelle du dispositif, rédiger un cahier des charges complet et finalement concevoir, simuler, tester et fabriquer l'aéroglisseur attendu.
	
	Ce projet sera également pour nous l'occasion de développer nos compétences linguistiques dans la langue anglaise par le biais de rapports oraux ponctuels lors des différentes étapes de développement du système. Les aspects linguistiques ciblés sont notamment la maîtrise du vocabulaire technique relatif au thème du projet ainsi que la maîtrise de formes classiques comme les formes passives. Nous serons également amenés à travailler avec des documents techniques et des ressources bibliographiques en langue anglaise tout au long du projet.
	
	Le projet intégrera finalement les aspects humains de gestion de projet et de communication orale et écrite. Nous ferons vérifier dans un premier temps nos lettres de motivation et CV, pour ensuite définir une stratégie de projet basée autour d'un planning de travail. Les présentations orales de pré-étude, de définition de solution, d'étude... seront réalisées en salle banalisée et filmées afin d'obtenir un meilleur débriefing sur nos méthodes de communication et sur l'amélioration de celles-ci.
	
	%Ajouter phrase confinement et distanciel