\usepackage[utf8]{inputenc}
\usepackage[T1]{fontenc}
\usepackage[french]{babel}
\usepackage[top=20mm, bottom=25mm, left=15mm, right=15mm]{geometry}
\usepackage{hyperref}
\usepackage{mathpazo}
\usepackage{multicol}
	
\usepackage{color}
	\definecolor{apricot}{rgb}{0.98, 0.81, 0.69}
	\definecolor{bisque}{rgb}{1.0, 0.89, 0.77}
	\definecolor{champagne}{rgb}{0.97, 0.91, 0.81}
	\definecolor{cobalt}{rgb}{0.0, 0.28, 0.67}
	\definecolor{burntumber}{rgb}{0.54, 0.2, 0.14}
	
\usepackage{listings}
	\lstset{	
		basicstyle={\scriptsize},
		language=C,
		numbers={left},
		stepnumber=2,
		numbersep=5pt,
		inputencoding={latin1},
		backgroundcolor=\color{champagne},
		frame=l,
		numberstyle=\tiny\color{burntumber},
		breaklines=true,
		xleftmargin=10pt,
		framexleftmargin=1pt,
		keywordstyle=\bfseries\color{green!40!black},
	 	commentstyle=\itshape\color{burntumber},
	  	identifierstyle=\color{cobalt},
	 	stringstyle=\color{orange},
	 	consecutivenumbers=false,
	 	multicols=2
	 	}
	\renewcommand{\lstlistingname}{Code}
	\renewcommand{\lstlistlistingname}{Table des extraits de code}
	
\usepackage{graphicx}
\usepackage{pgfplots}
\pgfplotsset{compat=1.15}

\usepackage[american voltage, american current, americanresistors]{circuitikz}
\usetikzlibrary{babel}

\usepackage{amsmath}
\usepackage{amssymb}
\usepackage{mathrsfs}

\usepackage{fancyhdr}

\setcounter{secnumdepth}{3}

\newcommand{\insertcode}[3]{
\begin{center}
	\vbox{ #3
		\lstinputlisting[language=C, linerange={#2},caption={[Code extrait de \lstname: Ligne #2] \textbf{ }}]{#1}
	}
\end{center}
}

\pagestyle{fancy}
\fancyhead[L]{Projet GE2}
\fancyhead[C]{\textit{Aéroglisseur radiocommandé}}
\fancyhead[R]{Étude - Mai 2020}
\fancyfoot[L]{INSA Strasbourg}
\fancyfoot[R]{Tristan DRUSSEL - Florian POUTHIER}
\headheight=15.5pt