	\chapter{Conclusion}
	
	%Ce rapport de pré-étude a pour objectif de présenter le thème du projet GE II ainsi que les objectifs à atteindre en vue de proposer un système fonctionnel et homologable pour la compétition finale en fin de semestre. Nous y exposons une analyse fonctionnelle du système à réaliser, un cahier des charges détaillé ainsi qu'un développement des sous-fonctions du dispositif à concevoir.
	Ce rapport d'étude final résume nos travaux réalisés sur le semestre entier.\\
	
	La conception d'un aéroglisseur radiocommandé est un projet d'envergure faisant appel à des notions d'électronique de puissance comme d'électronique numérique. Chaque membre du binôme ayant un attrait plus important pour l'une des deux matières, nous nous sommes répartis la charge de travail par domaine de connaissances. Un des membres du binôme a développé la partie liée à la communication et au contrôle du système tandis que l'autre membre du binôme s'est concentré sur l'onduleur pour le moteur Brushless DC ainsi que la génération des différents niveaux de tension nécessaires au fonctionnement du dispositif.\\
	
	Ce projet, bien que n'ayant pas pu être réaliser de manière conventionnelle, nous a permis de développer de nouvelles compétences et de nouvelles connaissances. En effet, le projet étant transdiciplinaire il faut avoir des notions dans différents domaines afin de comprendre son travail. Par exemple, pour programmer le pilotage de l'onduleur, il faut d'abord s'être intéressé à comment le moteur doit être alimenté.\\
	
	De plus, la réalisation de ce projet en distanciel à inciter à travailler différemment, on ne pouvais pas, par exemple, rejoindre son binôme pour lui demander quelque chose sur un point précis de son travail. Cela nécessite une bonne organisation, une bonne communication et de bon outil.\\
	
	Finalement malgré tous les événements, nous avons réussi à travailler dans de bonnes conditions et nous avons, à notre avis, réussi à remplir les principaux objectifs du projet. \\
	%Ce rapport de pré-étude va ainsi nous donner une solide base d'investigation pour le développement de solutions technologiques fiables et fonctionnelles dans la conception de notre système. Afin d'avoir un meilleur suivi de notre avancée, vous trouverez notre répertoire \href{https://github.com/tristanplouz/ProjetGE2}{github ici}.