	\chapter{Conclusion}
	
	%Ce rapport de pré-étude a pour objectif de présenter le thème du projet GE II ainsi que les objectifs à atteindre en vue de proposer un système fonctionnel et homologable pour la compétition finale en fin de semestre. Nous y exposons une analyse fonctionnelle du système à réaliser, un cahier des charges détaillé ainsi qu'un développement des sous-fonctions du dispositif à concevoir.
	
	%La conception d'un aéroglisseur radiocommandé est un projet d'envergure faisant appel à des notions d'électronique de puissance comme d'électronique numérique. Chaque membre du binôme ayant un attrait plus important pour l'une des deux matières, nous avons décidé de nous répartir la charge de travail par domaine de connaissances. Un des membres du binôme développera alors la partie liée à la communication Bluetooth, [...] tandis que l'autre membre du binôme se concentrera sur la commande du moteur Brushless DC ainsi que la génération des différents niveaux de tension nécessaires au fonctionnement du dispositif.
	
	%Ce rapport de pré-étude va ainsi nous donner une solide base d'investigation pour le développement de solutions technologiques fiables et fonctionnelles dans la conception de notre système. Afin d'avoir un meilleur suivi de notre avancée, vous trouverez notre répertoire \href{https://github.com/tristanplouz/ProjetGE2}{github ici}.